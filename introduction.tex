
%%%%%%%%%%%%%%%%%%%%%%%%%%%%%%%%%%%%%%%%%%%%%%%%%%%%%%%%%%%%%%%%%%%%%%%%%%%%%%%%
\section{Introduction}

At the turn of the 21st century, the Human Genome Project produced the first draft of the human genome sequence \cite{Lander:2001hk}. The technological developments that made this possible continued to evolve quickly, and the next-generation sequencing (NGS) technologies, also called high-throughput sequencing (HTS), enabled a myriad of ways to measure genomic and epigenomic phenomena. RNA-seq measures the vast transcriptome, chromatin immunoprecipitation followed by sequencing (ChIP-seq) measures patterns of gene regulation via protein-DNA interactions, and Whole Genome Bisulfite Sequencing (WGBS) measures the cytosines marked by methylation that are a crucial genomic regulatory mechanism. At the same time, advances in high-throughput metabolomics assays has enabled quantification of large numbers of the small molecules that define our metabolome. Indeed, metabolomics data serve as a direct signature of biochemical activity in organisms and is therefore easier to correlate with phenotype \cite{Patti:2012ek}. On this basis, metabolomics is a uniquely powerful tool in clinical diagnostics.

Each of these technologies produces a vast amount of data. The technologies probing the genome and epigenome yield tens of millions of short sequence reads, requiring computational and statistical methods to test hypotheses and make sense of them. Numerous algorithms and data structures have been developed to store, assess the quality of, and align the reads to a reference genome. Additional algorithms have been developed to determine the signal from the noise and determine differentially expressed genes \cite{Robinson:2010cw, Love:2014ka}, the location of transcription factor binding sites (TFBS) \cite{Zhang:2008gm}, the regions subject to histone modifications (HM) \cite{Xu:2014eh, Zhang:2014cu}, and methylated CpGs \cite{Krueger:2011eb}. In the realm of metabolomics, similar signal to noise issues require methods to determine chemical signatures representing small molecules present in samples \cite{Patti:2012ek}.

An important component in the analysis of data from each of the highlighted technologies is the ability to contextualize the information, facilitating interpretation. This can take the form of annotating TFBS, HMs, or methylated CpGs to genomic annotations to understand how these regulatory proteins and epigenetic marks are distributed and might affect gene regulation. At the same time, visualizing related data (such as percent methylation or fold change over background) across the annotations can help generate hypotheses. Another way to facilitate interpretation is to perform gene set enrichment testing to understand how the same regulatory proteins and epigenetic marks play roles in the context of established gene sets representing biological processes and pathways. Data integration can also help interpretation by helping to understand how two (or more) related measurements are changing with respect to each other. Lastly, the small molecules that make up our metabolome are better understood in their relationship to the other small molecules, and so visualizations representing this connectedness are useful for noticing relationships that can be more formally tested.

%%%%%%%%%%%%%%%%%%%%%%%%%%%%%%%%%%%%%%%%%%%%%%%%%%%%%%%%%%%%%%%%%%%%%%%%%%%%%%%%
\section{Background}

%%%%%%%%%%%%%%%%%%%%%%%%%%%%%%%%%%%%%%
\subsection{Histone modifications}

The human genome, when unfurled, extends to about 2m in length, but it is severely compacted into the nucleus of our cells. The beads-on-a-string model, whereby about 147bp of DNA is wrapped around an octomer of histones (forming a nucleosome), is accepted to be the primary structure enabling this compaction, and further compaction is possible with secondary and tertiary structures \cite{Luger:2012gc}. For decades, it has been known that the histones composing nucleosomes have polypeptide tails that contain post-translational modifications (PTMs), often called histone modifications (HMs).

Perhaps the most common and well-characterized forms of HMs include methylation and acetylation of lysines on H3, but a plethora of other chemical marks have been observed on other protein residues, and on histone tails of the other core histones \cite{Rothbart:2014kf}. Work across multiple laboratories over the previous decades has demonstrated that H3K27ac (acetylation of H3 at lysine 27) is associated with transcriptional activation, H3K4me3 is associated with active euchromatin and promoter regions, H3K9me3 is associated with DNA methylation and transcriptional repression, and H3K27me3 is also associated with transcriptional repression but is mutually exclusive with H3K9me3 \cite{Rothbart:2014kf}. These examples indicate a common theme, that the HMs of histones contribute to the complex state of chromatin, whether opened, closed, or poised. Indeed, the combination of such marks has been termed the 'histone code', which is still a topic under investigation \cite{Strahl:2000jj}.

Histone modifications can be measured genome-wide by using chromatin immunoprecipitation followed by sequencing (ChIP-seq) with an antibody aimed at the particular modification. The result of ChIP-seq experiments is millions of short sequence reads from the DNA fragments that were part of the nucleosomes selected for by the antibody. In other words, it is a small genomic region affected by the HM. The millions of short fragments, when aligned to a reference genome, compose a signal indicating regions subject to the HM under investigation. It is good practice for ChIP-seq to perform a corresponding control experiment with a non-specific antibody to determine the background pulldown rate of the antibody. A number of algorithms exist to compare the signal from the pulldown (IP or ChIP) against the control (often input DNA) \cite{Zhang:2008gm, Xu:2014eh, Zhang:2014cu} and call 'peaks' that represent the regions under the influence of that HM.

Understanding where such peaks fall by annotating them to genomic annotations is a common step in interpreting the experiment, and we describe a software package, annotatr, developed to accomplish this task in section \ref{intro:annotatr} and in detail in chapter \ref{chap4}. Another useful analysis is that of gene set enrichment on the HM peaks. In this analysis peaks are associated with genes, and a statistical test is performed to determine which sets of genes (representing biological processes and pathways) are enriched with signal from the HM. This part of the dissertation is described briefly in section \ref{intro:broad} and in detail in chapter \ref{chap2}.

%%%%%%%%%%%%%%%%%%%%%%%%%%%%%%%%%%%%%%
\subsection{DNA methylation}

The addition of a methyl group to the fifth position of genomic cytosine forms 5-methylcytosine (5mC), often called the fifth base, and is a widely studied epigenetic mark. In mammals, 5mC predominantly occurs in CpG context, but in other organisms it can occur in CHG and CHH contexts where H is an A,C, or T. 5mC is prevalent throughout various tissues with between 60\% - 80\% of CpGs being methylated \cite{Smith:2013cd}. 5mC has been implicated in various cellular processes such as transcriptional repression, X chromosome inactivation, embryonic development, genomic imprinting, alteration of chromatin structure, and transposon inactivation \cite{Yong:2016jw}.

A lesser understood epigenetic mark is formed by the oxidation of 5mC (catalyzed by TET proteins) creating 5-hydroxymethylcytosine (5hmC) \cite{Kriaucionis:2009bm, Tahiliani:2009kl}. Additional, iterative oxidation of 5hmC (also catalyzed by TET proteins) creates 5-formylcytosine (5fC) and 5-carboxylcytosine (5caC) \cite{He:2011hc, Ito:2011bv}, but we shall focus on 5mC and 5hmC in what follows. In general, 5hmC occurs at much lower rates than 5mC in the genome and varies in abundance across different cell types. In mouse, the central nervous system (CNS) contains the highest rates of 5hmC with up to 40\% that of 5mC. Heart and kidney are 25-50\% of CNS tissues, and spleen and thymus have 5-15\% of CNS tissues \cite{Wu:2017hu}.

The connection between 5mC and its oxidized forms lies in the active demethylation pathway discovered in \cite{Kriaucionis:2009bm, Tahiliani:2009kl}, and reviewed in \cite{Wu:2017hu}. The role of 5hmC as an intermediate in the active demethylation pathway would suggest that it is only a transient mark. However, a study wherein the turnover of oxidized 5mC was measured by isotope labeling suggests that 5hmC is a stable epigenetic mark \cite{Bachman:2014dla}. In addition, possible reader proteins have been found for 5hmC, which further strengthens the case that 5hmC is stable, and has biological roles differing from 5mC \cite{Spruijt:2013ep}.

There are many ways to measure DNA methylation and hydroxymethylation (reviewed extensively in \cite{Yong:2016jw} and \cite{Bock:2012kr}). In this work we focus on two classes of experiments: bisulfite conversion-based (BS) methods and immunoprecipitation-based (IP) methods. The BS class of methods rely on a bisulfite treatment of the DNA, which converts bare cytosines to thymine while methylated cytosines are protected. On this basis, one can determine which bases were methylated when compared against an \emph{in silico} bisulfite-treated reference genome. Whole Genome Bisulfite Sequencing (WGBS) and Reduced Representation Bisulfite Sequencing (RRBS) are the most widely used sequencing assays in this class. They result in base-resolution absolute quantification of methylation. The IP class of methods rely on an antibody that is specific to 5mC or 5hmC, and pulls down DNA fragments that have the corresponding mark. In particular the methylated DNA immunoprecipitation family of assays (MeDIP-seq for 5mC and hMeDIP-seq for 5hmC) are widely used. They result in region-resolution relative quantification of methylation.

While the BS methods described above are considered the gold standard (and are used widely by consortiums such as ENCODE and Roadmap Epigenomics), both methylation and hydroxymethylation protect cytosines from bisulfite-conversion, and so these assays actually measure 5mC \emph{and} 5hmC signal. In order to understand the possible distinct biological roles of 5hmC it is necessary to separate this signal. Two recent sequencing technologies have been developed to measure 5hmC, oxBS-seq and TAB-seq. The former has an oxidative step, whereas the latter has a glucosylation step followed by TET1 treatment. However, both are difficult to perform successfully and suffer from replicability issues, so neither platform is widely used, and there is a dearth of publicly available data. IP based methods can be used to determine 5hmC signal, and this data can be integrated with WGBS or RRBS to help separate the signal \emph{in silico}. This is precisely the motivation for developing the mint pipeline, briefly outlined in section \ref{intro:mint} and described in detail in chapter \ref{chap5}. Additionally, the genomic localization of 5hmC will be important to understanding its biological roles, and so the development of annotatr in chapter \ref{chap4} is also relevant.

%%%%%%%%%%%%%%%%%%%%%%%%%%%%%%%%%%%%%%
\subsection{Gene set enrichment}

Gene set enrichment (GSE) describes a group of methods that use biological knowledge bases, and statistical methods to functionally interpret the results of high-throughput experiments. The original GSE methods responded to the need to find biological meaning in long lists of differentially expressed genes resulting from the analysis of gene expression microarrays \cite{Subramanian:2005jt}. This approach has since been applied to RNA-seq \cite{Young:2010ud, Lee:2016es} and ChIP-seq \cite{McLean:2010iq, Welch:2014fb} data, as we will describe.

Concurrent to the development of microarrays was the debut of various biological knowledge bases such as the Gene Ontology \cite{Ashburner:2000ja} and KEGG Pathway database \cite{Kanehisa:2000jn}. These knowledge bases formalize the concept of a gene set on an evidentiary basis, and in the case of KEGG Pathway, enumerate the components of a biological process in the context of a network. Essentially, they contain our collective biological knowledge.

In the context of ChIP-seq data, GSE can only proceed once the peaks representing TFBSs or HMs are linked to genes. This is accomplished by what we call a gene locus definition, in other words, a genomic interval (or intervals) that can be considered to be regulatory regions of the gene. For example, a proximal promoter locus definition could be defined as the regions within 1 kilobase (kb) of a gene's transcription start site (TSS). As another example, a locus definition could be created by considering a region that extends halfway to the next gene's TSS or transcription end site (TES). Such a locus definition would essentially partition the genome and assign a peak to the nearest gene. Once ChIP-seq peaks are assigned to genes, the statistical test linking gene sets to the data can proceed.

The first statistical methods for GSE can be described as over-representation analysis, or the 2 $\times$ 2 table method, where genes are either differentially expressed or not (or in the case of ChIP-seq, have a peak or not), and genes are either in a gene set representing a biological process or not. A one-sided Fisher's Exact Test (FET) can then determine whether the gene set is overrepresented by genes having a signal. The DAVID enrichment tool uses this approach, though slightly modified by subtracting 1 from the cell containing the genes having signal and in the gene set \cite{Huang:2009gk}. However, as observed in a review of GSEA methods, the test used in DAVID, and other methods, make the assumption that all genes are equally likely to be differentially expressed, which is not necessarily true \cite{Khatri:2012fy}. In the context of ChIP-seq data, the assumption is that each gene is equally likely to have a peak. However, this is not true because genes with longer loci tend to be more likely to have a peak \cite{Taher:2009ko}. The ChIP-Enrich method was the first GSE method for genomic regions to account for this gene locus length bias on an empirical basis, and the correct type I error was demonstrated across dozens of ENCODE ChIP-seq data sets \cite{Welch:2014fb}. It's approach was that of a logistic regression model which evolved from the development of LRpath for gene expression microarray data \cite{Sartor:2008by}. We have developed a GSE approach specifically for histone modification ChIP-seq which we describe briefly in section \ref{intro:broad}, and in more detail in chapter \ref{chap2}.

%%%%%%%%%%%%%%%%%%%%%%%%%%%%%%%%%%%%%%
\subsection{Metabolomics}

Metabolism is defined as the chemical processes that occur within living organisms. A metabolite is a small molecule that is chemically transformed during metabolism, the metabolome is the collection of metabolites in organisms, and metabolomics is the study of the metabolome. Developments in mass spectrometry have enabled broad spectrum, high-throughput, measurement of thousands of metabolites simultaneously in biological samples. Metabolites are direct signatures of biochemical activity, making them easier to correlate with phenotype \cite{Patti:2012ek}. Correspondingly, metabolomics is of increasing interest for integration with genomic, epigenomic, and proteomic data. Metabolomic studies take two forms, targeted and untargeted assays. Targeted assays have the goal of detecting and measuring a single metabolite, or a selected group of metabolites. If a particular biological pathway is characterized, and of interest, the targeted approach makes sense. Untargeted metabolic assays are designed when the metabolites in a sample are not known. Such sasays can detect hundreds to thousands of metabolites simultaneously. Of the various mass spectrometry methods available, liquid-chromatography mass spectrometry (LC/MS) is considered the best approach for untargeted metabolomics experiments because it has the best coverage of different metabolite classes \cite{Buscher:2009ko}.

When performing an untargeted experiment with LC/MS, the result is intensity peaks representing the mass to charge ratio ($m/z$) and split by the retention time. Each of these peaks is called a 'metabolite feature', and algorithms have been developed to associate the metabolite features with known metabolite signatures. On this basis, LC/MS experiments detect metabolites in a sample, and are capable of detecting changes in metabolic signatures across different conditions.

Metabolomics is at the stage where experiments can detect changes in hundreds or thousands of metabolites, and this creates a need for functional interpretation. This is a similar situation that led to the creation of gene set enrichment analysis (as described above). Indeed, various metabolite set enrichment analysis (MSEA) methods exist \cite{Xia:2002gx, Xia:2010fv, Chagoyen:2011bl}, and they rely on established sets of metabolites from KEGG \cite{Kanehisa:2011fz}, the Human Metabolome Database (HMDB) \cite{Wishart:2012wa}, and the Chemical Entities of Biological Interest (ChEBI) \cite{Hastings:2012ww}, to name a few. However, there is currently no tool enabling the exploration of many metabolite set databases, which tests for overlaps of established metabolite sets, and which visualizes their relationships in heatmaps and networks. A tool for gene sets similar to this exists \cite{Sartor:2009fo}, and we believe that a similar tool for metabolite sets will enable researchers to better understand metabolites or metabolite sets of interest. We briefly describe ConceptMetab in section \ref{intro:metab}, and in more detail in chapter \ref{chap3}.

%%%%%%%%%%%%%%%%%%%%%%%%%%%%%%%%%%%%%%%%%%%%%%%%%%%%%%%%%%%%%%%%%%%%%%%%%%%%%%%%
\section{Dissertation overview}

The aforementioned high-throughput experiments generate an incredible amount of data, but that data must be put into context for it to be useful. This is true of gene expression data, epigenomics experiments such as those measuring transcription factor binding and histone modifications (ChIP-seq) or those measuring DNA methylation (WGBS, RRBS, etc.), as well as for metabolomics experiments that quantify small molecules (LC-MS). The field of transcriptomics had a head start compared to epigenomics and metabolomics, and consequently the tools for interpreting transcriptomics data are both more abundant and mature. Functional interpretation tools for epigenomics and metabolomics data are especially needed. While it is possible to use some approaches from transcriptomics in epigenomics and metabolomics, they always require modification to account for particular biases and properties of the data. In the chapters of this dissertation, I describe four tools I have developed that facilitate the functional interpretation of epigenomics and metabolomics data.

%%%%%%%%%%%%%%%%%%%%%%%%%%%%%%%%%%%%%%
\subsection{Chapter \ref{chap2}: Broad-Enrich}
\label{intro:broad}

In chapter \ref{chap2} we present a gene set enrichment method, Broad-Enrich \cite{Cavalcante:2014dr}, that is an extension of ChIP-Enrich \cite{Welch:2014fb}. Whereas ChIP-Enrich was designed primarily for narrow transcription factor binding site (TFBS) ChIP-seq peaks, Broad-Enrich is designed for broad histone modification (HM) ChIP-seq peaks.

The first step of gene set enrichment for ChIP-seq data is to assign peaks to genes. As described above, a gene locus definition is a way of defining regulatory control regions of genes. One can consider gene loci as consisting of the regions 1kb upstream and downstream of the TSSs, just the exons, or the gene bodies, to illustrate a few examples. Regardless, the standard practice has been to determine the intersection of a peak midpoint with a gene locus definition to assign the peaks to genes. However, we observed that HM peaks are wider and often span more than one gene's locus. Consequently, reducing a HM peak to its midpoint ignores potentially important regulatory information. In our enrichment model for Broad-Enrich, we therefore considered the ratio of a gene locus covered by peaks as the independent variable. As was observed in ChIP-Enrich, a bias is present that increases the probability of a peak occurring in longer gene loci. A similar bias must be accounted for in the Broad-Enrich model, because longer genes tend to have a lower proportion covered. The Broad-Enrich model empirically adjusts for this bias, as we demonstrate in detail in Chapter \ref{chap2}.

We demonstrate Broad-Enrich on 55 ENCODE HM ChIP-seq datasets, and show that the test achieves the correct type I error under the null hypothesis of no biological enrichment. Moreover, we demonstrate that the correction for the relationship between gene locus length and proportion of locus covered is necessary to achieve the correct type I error rate. We compare Broad-Enrich to Fisher's Exact Test and a binomial-test implemented in GREAT \cite{McLean:2010iq}, and show that Broad-Enrich finds more biologically relevant results and often with stronger enrichment signal. Finally, we explore how using a locus definition conforming to prior knowledge of where an HM tends to occur in a gene can improve the enrichment signal.

%%%%%%%%%%%%%%%%%%%%%%%%%%%%%%%%%%%%%%
\subsection{Chapter \ref{chap3}: ConceptMetab}
\label{intro:metab}

In chapter \ref{chap3} we present a metabolite database and exploratory tool, ConceptMetab \cite{Cavalcante:2016cj}, designed as a resource for querying the biological concepts associated with metabolites, as well as the relationships among biological concepts at the metabolite level.

Advances in mass spectrometry methods allow for higher throughput measurement of hundreds or thousands of metabolites. Consequently, experiments measuring changes in metabolites between conditions are becoming more common. However, there are not many tools that link metabolites to biologically meaningful concepts.

ConceptMetab is unique in its breadth of metabolite sets, which include biomedical concepts from KEGG, the three branches of the Gene Ontology (GO), and Medical Subject Headings from the National Library of Medicine (MeSH). In all, we annotated about 68,000 compounds to 16,000 biological concepts, and determined statistically significant associations among all possible combinations. The ConceptMetab web site allows users to explore the associations based on a metabolite or a biological concept of interest. Moreover, users can view supporting information and visualize relationships using network graphs and heatmaps. We demonstrated the utility of ConceptMetab with a few vignettes. Among them, to understand the molecular and anatomical risks and effects of atherosclerosis, to investigate the diseases associated with the unfolded protein reponse, and to explore the biological roles of a metabolite of interest.

%%%%%%%%%%%%%%%%%%%%%%%%%%%%%%%%%%%%%%
\subsection{Chapter \ref{chap4}: annotatr}
\label{intro:annotatr}

In chapter \ref{chap4} we present an R Bioconductor package, annotatr \cite{Cavalcante:2017gc}, designed to annotate genomic regions to genomic annotations that gives users the flexibility of selecting fine-grained annotations, as well as offering summarization and visualization options. A common step in genomic analyses is to annotate genomic regions to genomic annotations, such as genic features, CpG features, enhancers, etc. While a variety of tools exist to accomplish this task, we found that they suffer from some of the following shortcomings: 1) genomic annotations are too simple (e.g. annotating to gene bodies, without being able to distinguish between UTRs, exons, introns, etc.), 2) annotations are prioritized, meaning a genomic region can only be assigned to one annotation, which may not be the one most important to the researcher and ignores the possible importance of co-annotations in regulation, 3) an inability to visualize the annotations with covariate information (e.g. percent methylation for CpGs in different annotations), or 4) slow performance and/or high memory requirements.

With this state of affairs, we developed the R Bioconductor package annotatr to address each of these problems. We enumerate the variety of possible annotations from genic features and CpG features, long non-coding RNAs (lncRNAs), and enhancers. We also designed annotatr to report \emph{all} annotations intersecting a region because we consider it arbitrary to prioritize CpG islands over promoters, for example, when knowing a region falls in both simultaneously can be biologically important. We also demonstrate a variety of visualization functions in annotatr that enable users to explore data associated with the genomic regions across the annotations. This can be especially helpful to biologically interpret experiments. Finally, we demonstrate that annotatr is significantly faster than some of the alternative annotation packages.

%%%%%%%%%%%%%%%%%%%%%%%%%%%%%%%%%%%%%%
\subsection{Chapter \ref{chap5}: mint}
\label{intro:mint}

In chapter \ref{chap5} we present the methylation integration (mint) pipeline for analyzing, integrating, and annotating (with annotatr) DNA methylation and/or hydroxymethylation data [\emph{In press}]. As discussed above, the gold-standard technology for quantifying DNA methylation (WGBS) cannot distinguish between 5-methylcytosine (5mC) and 5-hydroxymethylcytosine (5hmC). However, current research indicates that 5hmC is a stable epigenetic mark that has different biological roles from 5mC. Consequently, differentiating the 5mC and 5hmC signals is important for understanding the different roles of the marks, as well as for understanding how corresponding changes  affect biological systems.

We describe the mint pipeline by analyzing a subset of an acute myeloid leukemia (AML) public dataset containing cancer samples with mutations in the IDH2 gene and normal bone marrow (NBM) samples. Previous findings indicate that mutations in IDH2 lead to increased 5mC levels and decreased 5hmC levels, caused by an inhibition of the active demethylation process. In brief, we describe the following modules used in mint: 1) the alignment module, which does initial QC steps, read trimming and alignment, 2) the sample module, which performs methylation quantification, 3) the comparison module, which tests for differentially methylated CpGs or regions with multi-factor designs with covariates, and 4) the integration module, which segments the genome into regions of 5mC / 5hmC or differential 5mC / 5hmC, depending on the experimental design. We also describe the variety of visual outputs of each module, which include genomic annotations and a UCSC Genome Browser track hub which enables users to view their data with more biological context to generate hypotheses and better understand their experimental results.
