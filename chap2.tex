\noindent This work has been published as: \textbf{R. G. Cavalcante}, C. Lee, R. P. Welch, S. Patil, T. Weymouth, L. J. Scott, and M. A. Sartor, "Broad-Enrich: functional interpretation of large sets of broad genomic regions.," \emph{Bioinformatics}, vol. 30, pp. i393–-400, Sept. 2014.

%%%%%%%%%%%%%%%%%%%%%%%%%%%%%%%%%%%%%%%%%%%%%%%%%%%%%%%%%%%%%%%%%%%%%%%%%%%%%%%%
\section{Introduction}
\label{broadenrich_introduction}

Chromatin immunoprecipitation followed by high-throughput sequencing (ChIP-seq) identifies transcription factor (TF) binding sites and the locations of histone modifications (HM) across the genome \cite{Barski:2007gh}, and is a step toward better understanding the gene regulatory programs of living organisms. Numerous algorithms, termed peak callers, have been developed to detect the genomic regions of significant signal (peaks) within the millions of aligned reads resulting from ChIP-seq experiments \cite{Valouev:2008ev, Zhang:2008gm, Bailey:2013ie}. Some of these peak callers are geared specifically to HMs, which are known to exhibit broader enriched domains on average compared to TFs \cite{Zang:2009ht}. HMs are numerous and varied, and like TFs, often drive the regulation of a specific biological program, such as cellular differentiation \cite{Sen:2008gx} or growth \cite{Bernstein:2006ip}. Specific signatures often occur at HM intersections, such as the bivalent domains observed for H3K4me3 and H3K27me3, which mark genes expected to be activated upon cellular differentiation \cite{Bernstein:2006ip, Pan:2007ey}. Other histone changes occur in disease progression \cite{Chi:2010bd} or in response to environmental signals \cite{Kaelin:2013kp}. Such signatures are likely often cell-type and context specific, and therefore assessing the biological commonalities among the targeted genes is a question of intense interest.

Gene set enrichment (GSE) is a common approach to infer biological function given a set of experimentally derived genes \cite{Draghici:2003uj}. GSE was originally developed to biologically interpret lists of differentially expressed genes derived from microarray studies \cite{Curtis:2005ck} in terms of particular biological functions, processes, or pathways (e.g., Gene Ontology \cite{Ashburner:2000ja} or KEGG Pathways \cite{Kanehisa:2000jn}). An early enrichment tool is DAVID \cite{Huang:2009gk}, which uses a slightly modified Fisher's exact test (FET) to determine whether experimentally derived genes significantly overlap a gene set representing a biological concept, relative to the remaining genes. Under the null hypothesis of no more overlap than expected by chance, FET assumes that each gene has the same probability of being detected as significant. In the context of GSE with ChIP-seq data, FET assumes that each gene has an equal probability of being associated with a peak. Although FET has been used with ChIP-seq data \cite{Blow:2010bu, Han:2013gv}, it is typically used only with peaks within or near gene promoters. When all peaks are used, the presence of a peak in a gene locus is often correlated with the length of the locus \cite{Ovcharenko:2005br}, thereby violating the FET assumption. We refer to this correlation as the locus length bias. Given that some gene sets contain genes that have, overall, significantly longer (e.g., nervous system, development, and transcription related) or shorter (e.g., metabolic processes and stimulus responses) than the average locus length, the possibility of confounding exists when no correction is made for locus length \cite{Taher:2009ko}. Using FET with only peaks near gene promoters removes nearly all of the length bias, but also ignores a large portion of the data.

Recent GSE tools for ChIP-seq experiments have attempted to correct for this length bias. One such tool, Genomic Regions Enrichment of Annotations Tool (GREAT), uses a binomial-based test to test whether the total number of peaks within the loci in a gene set is greater than expected relative to the total number of peaks, the total locus length of the gene set, and the non-gapped length of the genome \cite{McLean:2010iq}. In contrast to Fisher's exact test, the binomial test of GREAT assumes that the number of peaks in a locus and the locus length are proportional. Thus, FET and the binomial test have opposing assumptions regarding the relationship between the presence of a peak in a genomic region, and the length of that region. While FET is typically used after classifying each gene as either (a) having at least one associated peak or (b) having no peak, the binomial test uses the total number of peaks. Both methods typically use a single nucleotide point, the midpoint or mode of the peak, to represent the entire peak region.

We examined 100 TF and 55 HM ChIP-seq experiments from ENCODE \cite{ENCODEProjectConsortium:2012gc} for differences between peak sets from transcription-factor and histone based ChIP-seq experiments. HM peak sets have been observed to have broader peak regions than TFs, with individual peaks often spanning multiple genes \cite{Zang:2009ht}. We hypothesized that an enrichment method using such relevant regulatory information rather than simply the midpoint of each peak, as both Fisher's exact test and the binomial test do, would improve performance for HMs and other experiments resulting in broad domains.

To incorporate the properties of broad-domain peak sets into functional enrichment testing, we developed Broad-Enrich to functionally interpret large sets of broad genomic regions. A unique feature of our method is that we score gene loci according to the proportion of the locus covered by all peaks overlapping the locus, which we will refer to as the coverage proportion. Broad-Enrich then uses a logistic regression model that empirically adjusts for any bias in gene loci coverage relative to locus length, avoiding the pitfalls of either Fisher's exact test or binomial-based tests. We show that Broad-Enrich exhibits the correct type I error rate across 55 permuted ENCODE ChIP-seq datasets. We then illustrate the benefits of Broad-Enrich across the same set of 55 datasets, concentrating on H3K4me1,-2, and -3, H3K9me3, H3K27me3, and H3K79me2 in the GM12878 cell line.

%%%%%%%%%%%%%%%%%%%%%%%%%%%%%%%%%%%%%%%%%%%%%%%%%%%%%%%%%%%%%%%%%%%%%%%%%%%%%%%%
\section{Methods}
\label{broadenrich_methods}

%%%%%%%%%%%%%%%%%%%%%%%%%%%%%%%%%%%%%%
\subsection{Gene locus definitions}
\label{broadenrich_methods_locus}

We define a gene as the region between the furthest upstream transcription start site (TSS) and furthest downstream transcription end site (TES) for that gene. The UCSC knownGene table (human genome build hg19) was used to define TSS and TES sites. We removed small nuclear RNAs as they are likely to have different regulatory mechanisms than other genes and often reside within the boundaries of other genes. For functional enrichment testing we use three primary definitions of a gene locus (Figure \ref{chap2:fig:1}). (1) Nearest TSS: the region between the upstream and downstream midpoints of a gene's TSS and the adjacent gene's TSS; equivalent to assigning each peak to the gene with the nearest TSS.  (2) $\leq$ 5kb: the region within 5kb of all TSSs in a gene. If TSSs from the adjacent gene(s) are less than 10kb away, we use the midpoint between the two TSSs as the boundary of the locus for each gene. (3) Exons: the exons of each gene. When exons from multiple transcripts of the same gene overlap, the exons are consolidated into one continuous region. In the R package and on the website we include two additional definitions. (1) Nearest gene: the region from the midpoint between the TSS and the adjacent gene's TSS or TES (whichever is closest) to the midpoint between the TES and the adjacent gene's TSS or TES (whichever is closest). This is equivalent to assigning peaks to the nearest gene; (2) $\leq$ 1kb: same as $\leq$ 5kb, but within 1kb of all TSSs in a gene.

%%%%%%%%%%%%%%%%%%%%%%%%%%%%%%%%%%%%%%
\subsection{Proportional assignment of peaks to genes}
\label{broadenrich_methods_assignment}

A unique feature of Broad-Enrich is how peaks are assigned to gene loci. For a particular gene locus definition, each locus is scored according to the proportion covered by the union of all peaks overlapping the locus (Figure \ref{chap2:fig:1}). Our approach accounts for the extent to which a locus is covered by a peak, and allows coverage by multiple peaks.

%%%%%%%%%%%%%%%%%%%%%%%%%%%%%%%%%%%%%%
\subsection{Annotation databases}
\label{broadenrich_methods_databases}

Functional enrichment results presented here are performed on gene sets constructed from the Gene Ontology (GO) database and the KEGG Pathways database. We construct GO terms from GO biological processes, GO cellular components, and GO molecular functions using the org.Hs.eg.db and GO.db R packages. All analyses in the paper were performed using R version 3.0.1. KEGG Pathways are inherited from LRpath \cite{Kim:2012bk}. Eleven additional annotation databases are offered in the R package, including cytoband regions, Biocarta \cite{Nishimura:2001hd} and Panther pathways \cite{Mi:2013jj}, pFAM \cite{Punta:2012ko} and gene sets derived from literature-based Medical Subject Heading (MeSH) terms \cite{Kim:2012bk, Sartor:2009fo}. Prior to enrichment testing, all gene sets are filtered through the user selected gene locus definition so that only genes with a locus definition are included in the tests. By default, only gene sets containing between 10 and 2000 genes are tested. A minimum of 10 genes allows better convergence of the logistic regression model used for enrichment \cite{Peduzzi:1996cq} and the maximum of 2000 genes avoids general, less informative gene sets. Annotation databases were built for human (hg19), mouse (mm9 and mm10), and rat (rn4).

%%%%%%%%%%%%%%%%%%%%%%%%%%%%%%%%%%%%%%
\subsection{Broad-Enrich method for functional enrichment testing}
\label{broadenrich_methods_broadenrich}

We use a logistic regression framework to test for functional enrichment, similar to LRpath \cite{Sartor:2008by}, an enrichment testing method developed for microarray data. The independent variable $r$ for Broad-Enrich is the vector of proportions of each gene's locus that is covered by the union of all peaks (Figure \ref{chap2:fig:1} visually represents these proportions). The dependent variable is a binary vector indicating gene set membership (1 if the gene belongs to the gene set and 0 otherwise). Let $\pi$ be the proportion of genes in the gene set at a specified $r$ value and locus length $L$. Then the ratio $\frac{\pi}{1 - \pi}$ is the odds that a gene with peak coverage proportion $r$ and locus length $L$ is a member of a given gene set. If the log odds increases as $r$ increases, then we conclude the gene set is positively associated with the coverage proportion, and thus enriched with the experimental set of broad genomic regions. We use the model:
\begin{equation} \label{broadenrich_equation_model}
	\log \frac{\pi}{1 - \pi} = \beta_0 + \beta_1 r + SS (\log_{10} L)
\end{equation}
where $\beta_0$ is the intercept, $\beta_1$ is the coefficient of interest for the coverage proportion, the function $SS$ is a binomial cubic smoothing spline that adjusts for the potentially confounding effect of locus length, and the $log_{10}$-transformation is used to improve the model fit (data not shown).

The smoothing spline function is fit using generalized cross-validation to estimate the smoothing penalty, $\lambda$, and ten knots with the cubic spline basis as an approximation to a true cubic smoothing spline \cite{Wood:2010cl}. The overall model is fit using a penalized likelihood maximization approach with the gam function in the mgcv R package \cite{Wood:2010cl}. A Wald test is used to test the null hypothesis $H_0 : \beta_1 = 0$ versus the alternative $H_1 : \beta_1 \ne 0$ and to calculate the p-value for the significance of the coverage proportion coefficient, $\beta_1$ (Figure \ref{chap2:fig:1}). Gene sets with $\beta_1 > 0$ are enriched, while those with $\beta_1 < 0$ are depleted. P-values are corrected for multiple testing using the Benjamini-Hochberg false discovery rate adjustment \cite{Benjamini:1995cd}. For presented analyses, gene sets with $FDR < 0.05$ are considered to be significant.

%%%%%%%%%%%%%%%%%%%%%%%%%%%%%%%%%%%%%%
\subsection{Experimental ChIP-seq peak datasets}
\label{broadenrich_methods_datasets}

We used 155 ENCODE ChIP-seq datasets from 31 DNA binding proteins: 11 histone modifications (HMs) and 20 transcription factors (TFs) across 5 cell lines (GM12878, H1-hESC, HeLa-S3, HepG2, and K562), representing the largest complete matrix of experiments of HMs and TFs among tier 1 and tier 2 cell lines. Peaks for the 55 HM datasets were called by the ENCODE Consortium using Scripture, and used as is. The 100 TF datasets were originally called using a variety of peak callers according to the lab of origin. We implemented a standard peak calling pipeline for the TF datasets (Section \ref{broadenrich_methods_peaks}).

%%%%%%%%%%%%%%%%%%%%%%%%%%%%%%%%%%%%%%
\subsection{Standard peak calling pipeline}
\label{broadenrich_methods_peaks}
The 100 TF datasets used were originally called using a variety of peak callers according to the lab of origin. We implemented a standard peak calling pipeline for the TF datasets by downloading the alignments for each replicate and corresponding controls (including control replicates when present), calling peaks using MACS2 (\url{https://github.com/taoliu/MACS/}), and using the Irreproducible Discovery Rate (IDR) approach to combine peak information across the replicates \cite{Li:2011gg}. Briefly, the IDR approach determines the optimal number of peaks to select from the ranked pooled-replicate peak set based on a model of the reproducibility of peaks between the replicates.

We followed the recommendations of the ENCODE Consortium in our implementation of the IDR pipeline \cite{Landt:2012cl}. MACS2 was run using pooled controls on biological replicates, pooled pseudo-replicates, and biological-pseudo replicates with a p-value threshold of 1e-03, up-scaling the smaller dataset to match the larger dataset as recommended, and otherwise default settings. IDR rates were dynamically chosen depending on the number of peaks called for the biological replicates and the pooled pseudo-replicates.

%%%%%%%%%%%%%%%%%%%%%%%%%%%%%%%%%%%%%%
\subsection{Power study comparing Broad-Enrich to Fisher's exact test}
\label{broadenrich_methods_power}
For the 16 datasets with correct type I error for Fisher's exact test (Tables \ref{chap2:table:1}), we designed a simulation study to assess the power of Broad-Enrich versus FET. We randomly selected small, medium, and large GO terms: GO:0007435 'salivary gland morphogenesis' (30 genes), GO:0009306 'protein secretion' (150 genes), and GO:0048878 'chemical homeostasis' (763 genes). We simulated the proportion of genes in these terms with peaks (0.25, 0.5, 0.75, or 0.9), and in the case of Broad-Enrich, simulated the proportion of each gene locus covered by a peak (0.25, 0.5, 0.75, or 0.9). FET uses a 0/1 binary measure of locus/peak relationship. We performed 50 simulations for each combination of variables, and each GO term, and HM dataset. True positives are counted as gene sets with p-value $< 0.05$.

%%%%%%%%%%%%%%%%%%%%%%%%%%%%%%%%%%%%%%
\subsection{Permutations to test type I error rate}
\label{broadenrich_methods_permutations}

Two permutation scenarios were performed to assess the type I error rate of the enrichment tests under the null hypothesis of no true biological enrichment with gene sets from GO. In both scenarios, gene labels are permuted so that each gene is given the GO term assignments of a randomly chosen gene. Preserved in both scenarios is the number of genes in a gene set, and the correlations among the gene sets inherited from their parent/child relationships.

In the first scenario (referred to as 'Permuted'), we randomly permute gene labels relative to locus length and peak coverage proportion. The resulting permutations remove true biological association and remove the locus length bias inherent in the GO terms. In the second scenario (referred to as 'Permuted in Bins'), gene labels are randomly permuted within bins of 100 genes sorted by locus length. This has the effect of preserving the relationship between locus length and peak coverage proportion in the dataset. The resulting permutations remove true biological association in the gene sets while maintaining any locus length bias. Tests exhibiting inflated type I error under this scenario in excess of the first scenario can be considered as not appropriately accounting for locus length. Each type I error estimate was based on 5,404 tests.

%%%%%%%%%%%%%%%%%%%%%%%%%%%%%%%%%%%%%%
\subsection{Alternative functional enrichment testing methods}
\label{broadenrich_methods_alternatives}

We compared the functional enrichments for the 55 HM experiments (11 HMs across 5 cell lines) found with Broad-Enrich to those found by Fisher's exact test and our implementation of the binomial test of GREAT \cite{McLean:2010iq}. Additionally, we determined the type I error rate for a simplified version of the Broad-Enrich model excluding the smoothing spline (simple logistic regression (LR) model) to assess its necessity. Genes that were annotated in GO or KEGG, and had a defined locus were included in the analyses. We used a two-sided Fisher's exact test to test for association of peak presence ($\geq 1$ peak midpoint within a gene locus) and gene set membership. We used a binomial test similar to the one described in GREAT; we calculate the probability of seeing greater than or equal to the number of peaks we observe for a gene set, $\pi$, with the formula:

\begin{equation} \label{broadenrich_equation_great}
	\sum_{i = k_{\pi}}^{n} {n \choose i} p_{\pi}^i (1 - p_{\pi})^{n - i}
\end{equation}

where $n$ is the total number of peaks within gene loci in any gene set, and $k$ is the number of peaks annotated to gene set $\pi$. The term $p$ is defined as the expected proportion of peaks in gene set $\pi$. In other words, $p$ is the total non-gapped gene loci length in the gene set, divided by the total non-gapped length of loci with at least one gene set annotation. P-values are calculated as the probability of observing k or more peaks in the gene set.

We also used GREAT (\url{http://bejerano.stanford.edu/great/}) with hg19, the non-gapped genome as the background region, and the single nearest gene within 9999kb association rule excluding curated regulatory domains.

%%%%%%%%%%%%%%%%%%%%%%%%%%%%%%%%%%%%%%%%%%%%%%%%%%%%%%%%%%%%%%%%%%%%%%%%%%%%%%%%
\section{Results}

%%%%%%%%%%%%%%%%%%%%%%%%%%%%%%%%%%%%%%
\subsection{Differences between histone and transcription-factor based ChIP-seq data}
\label{broadenrich_results_differences}

We examined peaks from 155 ENCODE ChIP-seq experiments including 20 transcription factors and 11 histone modifications in 5 cell lines. We find that, relative to transcription factor based experiments, ChIP-seq experiments detecting histone modifications tend to have more peaks, broader peaks, and more variable peaks widths (Figure \ref{chap2:fig:2}). We also find histone based peaks tend to cover a much larger percentage of the hg19 genome (Figure \ref{chap2:fig:2}).

In addition to more and broader peaks in the HM datasets, we observed that the HM datasets also tend to have a higher proportion of peaks intersecting two or more gene loci compared to TF datasets. With the nearest TSS locus definition, we find the percentage of peaks covering two or more gene loci tends to be higher for HMs (median = 5.78\%, range = 1.71\%-24.66\%) than for TFs (median = 2.64\%, range = 0.17\%-8.82\%) (Figure \ref{chap2:fig:2}). Similarly, the percentage of peaks covering three or more loci is higher for HMs (median = 0.60\%, range = 0.17\%-7.64\%) than for TFs (median = 0\%, range = 0.00\%-0.14\%) (Figure \ref{chap2:fig:2}). The properties observed in HM peak sets indicate current methods may be ill-suited for detecting functional enrichment in HM ChIP-seq data.

%%%%%%%%%%%%%%%%%%%%%%%%%%%%%%%%%%%%%%
\subsection{Broad-Enrich method}
\label{broadenrich_results_method}

Based on the differences observed between transcription factors and histone modifications in ChIP-seq data, we aimed to develop an enrichment testing method that accounts for the extent to which each histone modification is associated with each gene. Using the number of peaks associated with a gene, as GREAT does, would yield stronger association to a gene with two very narrow peaks than to a gene with one very broad region covering the entire gene. Using a binary indicator of whether a gene has at least one peak associated with it, as is done with Fisher’s exact test (FET), would not account for any differences in the proportion of the gene locus covered.  Both approaches ignore instances where a peak covers a significant portion of the loci of two or more genes.

We first define the gene locus definitions, which capture the main trends of where HMs tend to occur relative to exons and TSSs. In this paper, we use (1) the region(s) within 5kb of every TSS of a gene ($\leq$ 5kb), (2) the combined exon regions for a given gene (exons), and (3) the region between the upstream and downstream midpoints between a gene’s TSS and the adjacent gene’s TSS (nearest TSS) (Figure \ref{chap2:fig:1}). These locus definitions represent binding in the greater promoter regions, throughout gene bodies, and anywhere in the surrounding genomic region including enhancers (assigned to the gene with the nearest TSS), respectively.

Given a locus definition, the proportion of each gene locus covered by all peaks overlapping the locus is determined. To test for significant enrichment, we use a logistic regression approach with gene set membership as the outcome and the proportion of a locus covered as the predictor. Due to the known confounding effect of locus length relative to the presence of ≥1 peak \cite{Taher:2009ko}, we examined and observed a similar relationship between locus length and peak coverage proportion (Figure \ref{chap2:fig:3}). We correct for $log_{10}$ locus length empirically using a binomial cubic smoothing spline (see Methods for more detail). P-values are then calculated for enrichment, and adjusted for multiple testing.

Broad-Enrich outputs three tab-delimited text files: (1) peak-to-gene locus assignments from the input peak set with lengths of peaks, loci, and overlap; (2) the gene locus coverage information after aggregating over all peaks overlapping a locus; (3) the enrichment results, with significance values and summary information for tested gene sets. QC plots showing the relationship between log10 locus length and the proportion of the locus covered by a peak are also output (Figure \ref{chap2:fig:3}).

%%%%%%%%%%%%%%%%%%%%%%%%%%%%%%%%%%%%%%
\subsection{Investigation of type I error}
\label{broadenrich_results_error}

Under the null hypothesis of no true gene set enrichment, the type I error rate, or proportion of false positives, for a dataset at a given threshold α is the proportion of gene sets with p-value less than $\alpha$. A method with type I error rate higher than the expected α level will result in an overabundance of false positives. We investigated the type I error rates for Broad-Enrich, the simple LR model, the binomial-based test, and FET, for 55 HM datasets under two permutation scenarios using the nearest TSS locus definition. Both permutations remove any true biological association between gene sets and the genes they contain. The first scenario (Permuted) assesses type I error of the enrichment test under no locus length bias. The second scenario (Permuted in Bins) has the effect of preserving the locus length properties of the gene sets, and illustrates the extent to which the type I error rate is affected by locus length.

We find that Broad-Enrich exhibits the correct type I error rates in both permutation scenarios and at different $\alpha$ levels. The binomial test exhibits severely inflated type I error in both scenarios, and both the simple LR model and FET exhibit the correct type I error rate in the Permuted scenario, but have inflated error for the Permuted in Bins scenario (Figure \ref{chap2:fig:4} ($\alpha = 0.05$) and ($\alpha = 0.001$), and Table \ref{chap2:table:1}). Comparing Broad-Enrich to the simple LR model, we conclude that the smoothing spline is essential for Broad-Enrich’s well-calibrated type I error. None of the 55 datasets tested exhibited correct type I error for the binomial-based test. Welch \emph{et al.} identified significant extra variability (beyond that expected by the binomial test) in the number of peaks assigned to genes in ENCODE ChIP-seq data; they show this, together with the incorrect assumption of the binomial test with respect to locus length  accounts for the inflated type I error \cite{Welch:2014fb}. In contrast, FET resulted in correct type I error for 16 of 55 datasets under both permutation scenarios (Figure \ref{chap2:fig:4} and Table \ref{chap2:table:1}). The inflated type I error of the remaining 39 datasets results from FET being unable to account for the locus length bias present in these datasets \cite{Welch:2014fb, Taher:2009ko}. We compare the enrichment results for these 16 datasets to those of Broad-Enrich in Section \ref{broad_results_great}.

%%%%%%%%%%%%%%%%%%%%%%%%%%%%%%%%%%%%%%
\subsection{Summary of ENCODE histone modification enrichment results}
\label{broadenrich_results_encode}

We tested for gene set enrichment using Broad-Enrich in the same 55 HM ChIP-seq datasets from the ENCODE Consortium. We find that significantly enriched gene sets outnumber significantly depleted gene sets by about 3:1 over all the datasets (Table \ref{chap2:table:2}). The number of enriched gene sets varies greatly among experiments, with as few as 8 for H3K9me3 in K562 and as many as 1,058 for H3K4me2 in H1-hESC (median number of enriched gene sets = 664) out of 5,591 total gene sets tested from GO and KEGG, and using the nearest TSS locus definition. For a fixed histone, the number of enriched gene sets can vary greatly across the 5 cell lines (e.g. H2az range = 74-767 and H3K9me3 range = 8-253) suggesting different biological activity for such HMs across GM12878, H1-hESC, HeLa-S3, HepG2, and K562.

For each histone modification we determined the extent of overlap among significantly enriched gene sets across the 5 cell lines with the nearest TSS locus definition (Table \ref{chap2:table:3}). GM12878 and H1-hESC tend to have the highest percentage of unique enrichments across all HMs. This could be an indication of more specific regulation via histone modifications in these cell lines compared to the others. H3K36me3 and H3K79me2 exhibit the highest percentage of enriched gene sets common to all cell lines (39\% each). Both modifications tend to occur within the gene body, and the observation of many mutually enriched gene sets could be a result of their necessary functions in constitutively expressed gene groups required by cells, such as transcription and RNA processing \cite{ENCODEProjectConsortium:2012gc}. H2az had the smallest percent (0.1\%) of mutually enriched gene sets among all five cell lines, with the most uniquely occurring in the embryonic stem cell line.

%%%%%%%%%%%%%%%%%%%%%%%%%%%%%%%%%%%%%%
\subsection{Comparison of Broad-Enrich to Fisher’s exact test and GREAT}
\label{broad_results_great}

FET has an acceptable type I error rate ($\leq 0.05$ at $\alpha = 0.05$ level) in only 16 out of 55 datasets (Figure \ref{chap2:fig:4} and Table \ref{chap2:table:1}). These datasets tend to have fewer peaks overall, and more peaks located within 5kb of the TSS compared to the 39 HM datasets with type I error rate $> 0.05$. For each of these 16 datasets, we compared the average peak coverage proportion of gene loci in the gene sets uniquely enriched by Broad-Enrich to those uniquely enriched by FET. The gene sets uniquely enriched by Broad-Enrich have a consistently higher proportion of the gene locus covered (Table \ref{chap2:table:4}). We also examined the percentage of significant enrichments which were stronger in one method versus the other by comparing the FDR values of gene sets enriched in either method. Broad-Enrich resulted in stronger enrichment signal in 12 of 16 datasets (Table \ref{chap2:table:4}). Finally, we compared the power of Broad-Enrich to FET in the 16 datasets by varying the proportion of genes with a peak, and the proportion of each gene locus covered by a peak. We find that Broad-Enrich has higher power than FET in nearly all cases (Table \ref{chap2:table:5}).

For comparison with GREAT (v1.8.2), we selected 6 histone datasets (H3K4me1,-2,-3, H3K9me3, H3K27me3, and H3K79me2 in the cell line GM12878) representing a mixture of activators/repressors and binding close/distal to TSSs. We tested all GO terms using the “single nearest gene” within 9999kb gene regulatory domain definition provided in GREAT because it is most similar to the nearest TSS definition in Broad-Enrich. We compared relative ranks of enrichments since the binomial-based test implemented in GREAT has overly significant p-values (inflated type I error rate). Comparing the top 20 ranked GO terms for each enrichment test, we find that compared to GREAT, Broad-Enrich consistently finds gene sets with higher coverage in terms of the proportion of each gene locus having the HM (Table \ref{chap2:table:6}).

The GM12878 cell line is a lymphoblastoid cell line. Lymphoblasts are naïve lymphocytes, which is the term used for any of the 3 types of white blood cell (leukocytes) in the vertebrate immune system. H3K4me1 is a known general transcriptional activator. The top 20 ranked GO terms for H3K4me1 in Broad-Enrich include leukocyte activation, lymphocyte activation, regulation of lymphocyte activity, positive regulation of immune response, and regulation of leukocyte activation (Tables \ref{chap2:table:7}, \ref{chap2:table:8}, and \ref{chap2:table:11}A). None of the above (and only one immune-related term) are in the top 20 ranked GO terms according to GREAT. In contrast, the top terms ranked by GREAT included mitochondrion and ribonucleotide binding related gene sets, which are not as strongly related to the known properties of GM12878 (Tables \ref{chap2:table:7}, \ref{chap2:table:8}, and \ref{chap2:table:11}B).

H3K27me3 is a known repressor of differentiation and developmental genes. Within the top 20 ranked GO terms from Broad-Enrich we find tissue development, organ morphogenesis, epithelium cell differentiation, and regionalization. According to GREAT, none of the above or related GO terms are ranked in the top 20, and only one is in the top 100 (Tables \ref{chap2:table:9} and \ref{chap2:table:10}). Moreover, the top terms ranked by GREAT included metabolic processes and energy/transport related gene sets, which are not commonly associated with the regulatory targets of H3K27me3.

In both instances we find that the binomial test not only finds an overabundance of significant ($FDR < 0.05$) terms, as indicated by its inflated type I error rate, but also that Broad-Enrich ranks biologically relevant terms better than GREAT.

%%%%%%%%%%%%%%%%%%%%%%%%%%%%%%%%%%%%%%
\subsection{Effect of locus definition on enrichment}
\label{broad_results_locus}

It is known that some histone marks preferentially occur in particular locations relative to gene features. To investigate the effect of locus definition on enrichment signal, we ran Broad-Enrich for each of the 55 HM ChIP-seq datasets with the nearest TSS, exons, and $\leq$ 5kb locus definitions. We hypothesized that using a locus definition better conforming to the known genomic location of the histone mark would result in stronger enrichment signal.

H3K4me2, known to occur in promoters \cite{Pekowska:2010hk}, tends to have strongest enrichment signal with the $\leq$ 5kb locus definition across the five cell lines (Figure \ref{chap2:fig:5}). H3K4me3, also known to occur in promoters \cite{Bernstein:2006ip} shows results similar to H3K4me2 (not shown). H3K79me2 binds near the 5' end of gene bodies and overall we see the strongest enrichment signal when using the $\leq$ 5kb definition (Figure \ref{chap2:fig:6}). In contrast H3K36me3 binds near the 3' end of the gene body and we see a somewhat stronger enrichment when using the exons definition compared to the $\leq$ 5kb definition (Figure \ref{chap2:fig:7}) \cite{Barth:2010jc, ENCODEProjectConsortium:2012gc}. Histone acetylation, such as H3K9ac, tends to occur near TSSs \cite{Barth:2010jc}, and we observe stronger enrichment signal for the $\leq$ 5kb locus definition across the five cell lines (Figure \ref{chap2:fig:8}). H3K27me3 gives stronger enrichment signal with the exons definition for all cell lines except H1-hESC, which performs best with the $\leq$ 5kb locus definition (Figure \ref{chap2:fig:9}). This may be indicative of a different regulatory regime for H3K27me3 in embryonic stem cells versus the other cell lines, consistent with current literature \cite{Xie:2013fk}. H3K4me1 is considered a distal, activating mark \cite{Dong:2012di}, and exhibits stronger enrichment signal with the nearest TSS locus definition in GM12878 and HepG2 but stronger signal with $\leq$ 5kb in H1-hESC, HeLa-S3, and K562 (Figure \ref{chap2:fig:10}). Broad-Enrich results from the additional tier 2 ENCODE cell lines A549, Huvec, and Monocytes-CD14+, and using the same three locus definitions, resulted in the same overall conclusions for the 11 HMs above (not shown). Overall, we observed that the locus definition closest to the known locations of an HM provided the strongest enrichment results. These results should be interpreted in light of the fact that nearest TSS is the only locus definition to include all peak regions; thus important information about individual genes within enriched gene sets may be lost for the $\leq$ 5kb or exons definitions.

%%%%%%%%%%%%%%%%%%%%%%%%%%%%%%%%%%%%%%%%%%%%%%%%%%%%%%%%%%%%%%%%%%%%%%%%%%%%%%%%
\section{Discussion}
\label{broadenrich_discussion}

Functional enrichment testing leverages our collective biological knowledge together with high-throughput genomic technologies in a statistical framework to functionally interpret new biological data. Unique properties observed in ChIP-seq data for histone modifications have led to the use of specialized peak calling algorithms. These properties, combined with the bias observed in gene loci coverage relative to locus length present challenges to existing functional enrichment methods. We have developed Broad-Enrich to address these issues in functionally interpreting large sets of broad genomic regions. Our approach uses the proportion of a gene locus covered by all peaks overlapping the locus, and a correction accounting for the locus length in a logistic regression model with gene set membership as the outcome.

Inflated type I error rates result in an overabundance of false positive results, while well-calibrated type I error rates result in accurately reported false discovery rates. We demonstrate that Broad-Enrich has a well-calibrated type I error rate across 55 HM ChIP-seq datasets representing a wide variety of technical and biological characteristics. In contrast, the binomial-based test consistently exhibits inflated type I error, while Fisher’s exact test (FET) has the correct type I error for only 16 of the 55 datasets. These 16 HMs represent transcriptional activators, or HMs occurring in actively transcribed genes. Even for these 16 HMs, Broad-Enrich tends to provide stronger enrichment signal than FET. Compared to GREAT, Broad-Enrich finds more biologically relevant terms in the top ranked gene sets, as illustrated with immune function related terms for  H3K4me1 and H3K27me3 in the context of lymphoblastoid cell line GM12878. While rank comparisons are not ideal, in the absence of a gold standard, we rely on known biological roles for the HMs combined with known characteristics in cellular context.

Finally, we examined the effect of locus definition on the enrichment signal from Broad-Enrich. We see the strongest enrichment signal by using the locus definition closest to the known locations of the histone modification. For two HMs, we observe differences in the optimal locus definition. For H3K27me3 the exons locus definition performs best in all cell lines except for H1-hESC, where ≤5kb performs best. This difference could be explained by the role H3K27me3 plays in embryonic stem cells, where it is known to often occur in promoters of genes having CpG islands to regulate differentiation of ES cells \cite{Deaton:2011cz, Xie:2013fk}. For H3K4me1 we observe that nearest TSS performs best for GM12878 and HepG2, while $\leq$ 5kb performs best for the remaining cell lines. This might indicate that GM12878 and HepG2 cells rely more heavily on long-range enhancer activity for gene activation than the other three cell lines. These results emphasize that the definition with strongest enrichment signal tends to mirror the currently understood location of HM binding. Our implementation of Broad-Enrich allows users to define their own custom locus definition to fit their own experimental contexts.

In addition to functionally interpreting single histone modification experiments, it is also possible to examine bi- or tri-valent HM signatures together (e.g. H3K4me3 and H3K27me3) with Broad-Enrich and compare the results to the HMs individually to determine if bivalency leads to unique biological function. Broad-Enrich is also applicable to other types of broad domain experiments, such as copy number variations.

As the regulatory programs of living organisms are better understood, Broad-Enrich may be improved with distal regulatory information from Hi-C experiments, allowing for more accurate locus definitions. The significance or strength of each peak region reported by peak callers may also be incorporated in the enrichment model. Such future changes may bring functional interpretation of broad genomic regions closer to making optimal use of peak information.
