
\section{Conclusions}
\label{conclusions:general}

The maturation of high-throughput genomic, epigenomic, and metabolomic assays since the turn of the 21st century has enabled a multi-scale interrogation of basic research and clinical questions. The path from sequence reads and mass/charge ratios to knowledge requires an array of computational and statistical methods capable of performing quality control, separating signal from noise, testing hypotheses, and providing biological context. In this dissertation I have contributed to the field by building tools to functionally interpret epigenomic and metabolomic data, helping researchers better understand their experiments.

In Chapter \ref{chap2}, we introduced Broad-Enrich, a gene set enrichment tool designed specifically for ChIP-seq of histone modifications that uses a logistic regression model on the proportion of gene loci covered and corrects for a known bias related to the locus length of a gene. We demonstrated that Broad-Enrich has the correct Type I error rate across 55 diverse histone modification ChIP-seq experiments from the ENCODE project, whereas other tools such as Fisher's Exact Test and GREAT have inflated Type I error. The implication is that FET and GREAT return significant results even when no biological enrichment is present. We further demonstrated that the smoothing spline which corrects for the bias related to locus length is necessary for Broad-Enrich to achieve the correct Type I error. When comparing Broad-Enrich to FET using data sets with mutually correct Type I error, we find that in most cases Broad-Enrich has stronger enrichment signal. Moreover, by varying the proportion of genes with a peak and the proportion of each gene locus covered by a peak, we found Broad-Enrich has higher power than FET. Comparing Broad-Enrich to GREAT across six histone datasets from GM12878, we compared the relative ranking of gene set enrichments and found that Broad-Enrich finds more biologically relevant gene sets in the context of the lymphoblastoid cell line GM12878. Finally, we explored the effect of locus definition on Broad-Enrich results, and showed that selecting a locus definition according to prior knowledge of HM localization can lead to stronger enrichment results.

In Chapter \ref{chap3}, we introduced ConceptMetab, an interactive web tool for exploring metabolites and sets of metabolites. ConceptMetab leverages the KEGG Pathways to turn the familiar Gene Ontology into metabolite sets, which has not previously been done. Moreover, we leveraged previous work by colleagues to build a unique database linking metabolites to functions and diseases via the literature. In addition to building a database of metabolite sets, we calculated Fisher's Exact Test on all combinations of the sets to determine statistically significant overlap of metabolites. Users can explore the other metabolite sets with significant overlap in a variety of ways: as a table with summary information about the significance and number of overlaps (with links to display the metabolites in common), as a network where nodes are metabolite sets and edges represent signifcant overlap of metabolites, or as a heatmap to get a broad sense of which metabolites or groups of metabolites form the core of the intersection of many metabolite sets. We demonstrated the utility of ConceptMetab with a number of biological vignettes.

In Chapter \ref{chap4}, we introduced annotatr, an R package designed to annotate genomic regions to genomic annotations. We developed annotatr because existing tools were slower, less customizable, and lacking in visualization functions. In particular, we designed annotatr with a broad array of genomic annotations not available in many tools. In addition to standard genic and CpG island related annotations, we provide annotations to enhancers, chromatin states via chromHMM, lncRNA from GENCODE, and any data available in the AnnotationHub Bioconductor package. Importantly, annotatr returns all annotations for a region rather than one annotation according to a prioritization. This is an especially important feature because a region annotated to multiple annotations can help functional interpretation. Another feature unique to annotatr is its ability to visualize data associated with the genomic regions across the annotations. We demonstrated this feature with regions of differential methylation between two conditions. Together, the visualization and summarization functions included in annotatr provide an easy interface for users to explore their data, where tedious custom code would have been necessary.

In Chapter \ref{chap5}, we introduced mint, a flexible pipeline for processing, analyzing, integrating, and visualizing genome-wide 5mC and/or 5hmC data. The mint pipeline can use reads from one or many platforms, including bisulfite-conversion methods such as WGBS and RRBS measuring 5mC + 5hmC, and immunoprecipitation methods such as MeDIP-seq and hMeDIP-seq measuring 5mC and 5hmC, respectively. Quality control steps are performed from the outset, and summarized across all samples in a single web page. Reads are adapter and quality trimmed, and then aligned using an aligner appropriate for the sequencing platform. Methylation is quantified in the case of BS-based assays, and regions of methylation are found in the case of IP-based assays. Differential methylation can be determined under general design and with the use of numerical or categorical covariates. Finally, if an experiment is designed with the goal of integration, a genome segmentation is performed delineating regions of 5mC, 5hmC, both, or none on the basis of signal intersection. The results of most steps are annotated to genomic annotations to give biological context to the methylated and/or differentially methylated regions. Finally, a UCSC Genome Browser track hub is generated for the user to view sample-wise and comparison-wise data with any other data available from the UCSC Genome Browser. The mint pipeline is a powerful tool that automates the rather complicated task of analyzing DNA methylation and hydroxymethylation data from raw reads to integration and interpretation. It does so in a restartable and reproducible manner owing to its implementation in make, a well-established UNIX tool for handling complex workflows with file dependencies.

\section{Future Directions}
\label{future}

\subsection{Chapter \ref{chap2}: Broad-Enrich}
\label{broadenrich_conclusion}

There are essentially three ways to improve gene set enrichment (GSE) for genomic regions: 1) more accurately reflect the annotation of genes to biological processes and pathways, 2) more accurately reflect gene regulation represented by the locus definitions, and 3) use a model that maintains the expected type I error while improving the enrichment results in terms of biological relevance or increased power. In addition, GSE tools could be improved by the introduction of more interactive visualizations and diagnostic plots.

We have been working to more accurately reflect the regulation of genes captured by the locus definitions. In particular, we have been building and testing locus definitions that account for regulation from enhancers. The chipenrich R package includes relatively simple definitions accounting for regulation around the promoter (a fixed width around a TSS) and from within the coding elements of a gene body (exons and introns). In addition there are definitions such as 'nearest gene' and 'nearest TSS' that include the gene bodies, but extend beyond TSSs and TESs until the next TSS or TES. These definitions happen to allow for the possibility of enhancer regulation, but not explicitly by design, nor in an empirical way.

To explicitly account for enhancer regulation we are actively working on an approach connecting putative enhancer regions to their target genes. The putative enhancer regions are from chromHMM classifications \cite{Ernst:2012ii}, bi-directional nascent RNA transcription at non-annotated transcription start sites (TSS) with surrounding enhancer characteristics \cite{Andersson:2014bn}, and DNase hypersensitive sites \cite{Thurman:2012fe}). Enhancer regions can optionally be extended to capture more peaks for the downstream enrichment. The method of connecting enhancer regions to target genes uses both direct and indirect approaches. The direct approach is to use chromatin interaction analysis by paired-end tags (ChIA-PET) mediated by Pol2, Rad21 or CTCF (all proteins involved in enhancer-promoter DNA looping) to detect interacting chromatin regions. Indirect approaches include: 1) implicit regulation in CTCF mediated chromatin loops as described in \cite{Rao:2014eo}, nascent RNA transcription abundance correlation as in \cite{Thurman:2012fe}, and DNase hypersensitivity signal correlation as in \cite{Thurman:2012fe}. For enhancer regions that are not assigned by any of the aforementioned methods, we will build additional locus definitions where they are assigned to the gene with the nearest TSS.

The total number of combinations of 1) enhancer regions which are 2) extended or not extended, and 3) linked to target genes results in over one thousand possible combinations of enhancer definitions. Our first pass filter for which locus definitions to test further is based on the proportion of the genome covered by the definition, and the number of ChIP-seq peaks caught, on average, across dozens of ENCODE data sets. Those definitions with too high coverage, or too few peaks caught are undesirable.

Once desirable locus definitions are selected, we will evaluate the type I error rate of the corresponding enrichment tests, evaluate the biological relevance of their results, and compare the strength of enrichment to other locus definitions that could be construed to capture regulation by enhancers (nearest TSS or the regions further than 5kb from a TSS).

\subsection{Chapter \ref{chap3}: ConceptMetab}
\label{conceptmetab_conclusion}

ConceptMetab is first and foremost a database annotating metabolites to meaningful biological concepts and pathways. Uniquely, ConceptMetab includes metabolite sets related to diseases and many other Medical Subject Heading (MeSH) terms that are not available in other resources. A natural extension to ConceptMetab would be to allow users to provide a list of changed metabolites in an experiment (in PubChem or KEGG identifiers), and determine the sets in ConceptMetab which significantly overlap. Correspondingly, all of ConceptMetab's network and heatmap visualization tools would be available for the input set of changed metabolites. Current MSEA methods are still using Fisher's Exact Test \cite{LpezIbez:2016bt} combined with an FDR calculation for enrichment testing. ConceptMetab could easily be altered to run a user's list of changed metabolites using FET because the test is fast and easily implemented. One problem to deal with is the background set of metabolites to use. It has been noted that current metabolomics technologies can only detect 5-10\% of a sample's metabolome, but because the metabolite sets used for testing are based on experimental evidence which has the same limitations, this bias is thought to cancel out \cite{Xia:2010fv}. Another possible problem to consider is the presence of ubiquitous metabolites that appear in large numbers of sets (e.g. AMP and ATP). It may be desirable to weight their presence in inverse proportion to the number of metabolite sets they belong to in order to avoid too many false positives.

\subsection{Chapter \ref{chap4}: annotatr}
\label{annotatr_conclusion}

The annotatr package performs its task of annotating genomic regions to genomic annotations quickly and flexibly. In order to remain relevant, annotatr will need to keep current with new genome versions for the organisms it currently has. Moreover, the addition of more organisms, starting with the core model organisms would make the package more appealing to users outside of the human-mouse research axis. Another useful addition would be a function to query the table of annotations for those regions co-occurring in two desired annotations. This would quickly allow users to investigate methylation levels at CpG islands and promoters, for example, while knowing their genomic locations and gene information. This would be a companion feature to the useful visualization summarizing quantities over such co-occuring annotations (Figure \ref{chap4:fig:6}). Using our work in section \ref{broadenrich_conclusion}, we could also provide users with information about which genes are targeted by the enhancers their regions are annotated to. This will make the enhancer annotations more useful, and further help users interpret their data.

\subsection{Chapter \ref{chap5}: mint}
\label{mint_conclusion}

The mint pipeline streamlines the analysis of DNA methylation and hydroxymethylation data in a make framework that adds important components such as integration of 5mC and 5hmC signal, genomic annotations, and visualizations. The integration of RNA-seq gene expression data is of particular interest to understand how gene expression is changing in response to changes in 5mC and/or 5hmC. Similar to the approach of the methylation parts of the pipeline, we would use existing tools to start from raw RNA-seq reads and go through the analysis to differential expression, and then integrate the expression data with the methylation data. Given that our tests for differential methlylation (DSS and csaw) are capable of general designs with covariates, the same models could be used in edgeR \cite{Robinson:2010cw}, for example. Genes found to be differentially methylated would be cross-referenced with those found to be differentially expressed and the resulting table and corresponding quantities could be explored by the user. Moreover, additional visualizations could be generated. For example, the difference in methylation between groups across an annotation such as promoters could be plotted, separated by differential expression status. It has been observed that hydroxymethylation occurs at exon/intron boundaries \cite{Ehrlich:2014hj}, hinting at a possible role in differential splicing. Differentially methylated regions annotated to intron/exon boundaries (a built-in annotation in annotatr) could then be compared to differential isoform expression.

The mint pipeline could also be improved by annotating methlyation rates and differentially methylated CpGs or regions to transcription factor binding sites and aggregating methylation rates (or differences in methylation) within these sites. Doing so would enable users to investigate changed transcription factor binding between different groups, and would help predict which genes may be targeted by changes in methylation.

A technical change to mint that would increase usability and ease future maintenance would be a transition from make to snakemake. Snakemake is a Python variant of make, which allows the usage of any Python code in the execution of the pipeline. This enables the logic of the pipeline (i.e. which segments of the pipeline are run depending on the experimental setup) to be unified with the execution of the pipeline in a way that make does not easily allow. Moreover, snakemake is more easily run on high-performance computing clusters, which will make the mint pipeline more scalable as users include more samples in their analyses.

\section{Epilogue}

In this dissertation we have developed software tools to interpret data from epigenomics and metabolomics experiments. Each chapter embodies a different approach to facilitate this interpretation: Broad-Enrich focuses on interpreting broad genomic regions in terms of the pathways the regions may be regulating, ConceptMetab constructs a database for the exploration of biomedical concepts based on metabolites, annotatr provides genomic context for genomic regions with covariate data through annotation and visualization, and mint integrates data types and incorporates annotatr to help discern different roles for DNA methylation and hydroxymethylation. The integration of multiple data types will increase as the cost of omics experiments decreases, and this will necessitate robust tools capable of providing integrated context across the assays as well as integrative visualizations. The tools developed herein are initial steps in this crucial direction.
